\documentclass[conference]{IEEEtran}
\IEEEoverridecommandlockouts
% The preceding line is only needed to identify funding in the first footnote. If that is unneeded, please comment it out.
\usepackage{cite}
\usepackage{amsmath,amssymb,amsfonts}
\usepackage{algorithmic}
\usepackage{graphicx}
\usepackage{textcomp}
\usepackage{xcolor}
\usepackage{placeins}
\usepackage[hidelinks]{hyperref}
\usepackage{float}
\def\BibTeX{{\rm B\kern-.05em{\sc i\kern-.025em b}\kern-.08em
    T\kern-.1667em\lower.7ex\hbox{E}\kern-.125emX}}
\begin{document}

\title{Marketing Bancário}

\author{\IEEEauthorblockN{Arthur Abrahão Santos Barbosa}
\IEEEauthorblockA{\textit{Universidade Federal de Pernambuco} \\
\textit{Centro de Informática}\\
Pernambuco, Brasil \\
aasb2@cin.ufpe.br}
\and
\IEEEauthorblockN{Filipe Samuel da Silva}
\IEEEauthorblockA{\textit{Universidade Federal de Pernambuco} \\
\textit{Centro de Informática}\\
Pernambuco, Brasil \\
fss8@cin.ufpe.br}
\and
\IEEEauthorblockN{Nigel Mendes de Lima}
\IEEEauthorblockA{\textit{Universidade Federal de Pernambuco} \\
\textit{Centro de Informática}\\
Pernambuco, Brasil \\
nml@cin.ufpe.br}

}

\maketitle





\section{Objetivos}
\subsection{Objetivo Geral}
Através da análise de algumas informações referentes a um indivíduo, usando um classificador ingênuo de Bayes, prever se o mesmo irá se inscrever em um depósito a prazo. 
\subsection{Objetivos Específicos}
\begin{itemize}
\item 
\end{itemize}
\section{Justificativa}
Este projeto foi escolhido com base na maneira organizada e completa que o conjunto de dados  foi disponibilizado e por sua afinidade em aplicar-se os conceitos existentes, o banco de dados pode ser encontrando do site Machine Learning Repository, com o nome de "Bank Marketing Data Set"\cite{b1}.


Sua função é prover dados sobre a possibilidade de um cliente aderir ou não o serviço prestado pela agência com base em testes com múltiplas entradas de dados e com duas saídas possíveis, sim ou não. Seu público alvo são principalmente bancos, qualquer área de estudo sobre comportamento social e estudos sobre aprendizagem de máquina.
\section{Metodologia}

A partir de uma base de dados que contém informações sobre indivíduos que são clientes de um Banco, fazer o tratamento dos dados, isto é, através de separação entre dados utilizados para os experimentos e dados para o treinamento do classificador e analisar quantitativamente as informações contidas nos campos da base de dados, isto é, fazer uma análise exploratória desses dados. 
Fazer o treinamento do Classificador Ingênuo de Bayes a partir dos dados, e fazer uma validação. Na próxima etapa usar o classificador para responder a uma pergunta, para cada cliente que está nessa base de dados, "o cliente irá se inscrever em um depósito a prazo?", estes são os experimentos. Analisar os resultados qualitativamente e quantitativamente, o projeto será dividido nas seguintes etapas:
\begin{itemize}
\item \textbf{Pesquisa sobre o tema:}
Através da pesquisa blibliográfica, estudar  a relevância do assunto, e suas aplicações.
\item \textbf{Base de dados:}
A base de dados está disponível em \cite{b1}. Se refere a uma pesquisa de instituição financeira, possui 45211 instancias e cada instância possui 17 atributos.
\item \textbf{Tratamento dos dados:}
Fazer a limpeza e seleção dos dados que serão usados no projeto. Os dados selecionados serão divididos em dois grupos, dados para treinamento do classificador e dados para o experimento.
\item \textbf{Analise Exploratória:}
Através do Uso da biblioteca pandas, numpy e matplotlib, analisar a dispersão do conjunto de dados,
através de medidas de tendência central, desvio padrão e o plot de gráficos.

\item \textbf{Classificador Ingênuo de Bayes:}
	O Classificador Ingênuo de Bayes (Naive Bayes) considera em um experimento que as variáveis aleatórias dos campos são todas independentes, mesmo que na realidade elas não sejam, e mesmo assim ele possui uma boa precisão como modelo de classificador.
    \begin{itemize}
        \item \textbf{Implementação:}
        Através do uso da biblioteca scikit-learn, numpy e pandas, implementar um classificador ingênuo de Bayes
        para responder a seguinte pergunta: "o cliente irá se inscrever em um depósito a prazo?".
        \item \textbf{Treinamento do classificador:}
        A partir do conjunto dados previamente separados, treinar o classificador supondo que as vinte variáveis usadas são independentes.
        \item \textbf{Validação:}
        A partir do conjunto de dados restante, verificar se o classificador responde a pergunta corretamente e com que precisão.
    \end{itemize}
\item \textbf{Experimentos:}
Com o classificador em mãos, realizar alguns experimentos e verificar seus resultados.
\item \textbf{Análise dos resultados:}
A partir dos dados obtidos nas etapas anteriores, analisar os resultados obtidos.
\end{itemize}

\clearpage
\section{Cronograma de Atividades}
\begin{table}[!ht]
	\centering
    \begin{small}
        \begin{tabular}{cc}
        	\\
        	%\multicolumn{2}{c}{{\fontsize{13}{\baselineskip} \selectfont C}{\fontsize{11}{\baselineskip}\selectfont RONOGRAMA DE}{\fontsize{13}{\baselineskip} \selectfont A}{\fontsize{11}{\baselineskip}\selectfont TIVIDADES }}\\ 
        	\\
            \hline
            Data                    & Atividades\\
            \hline
            02/04/21                & Pesquisa Bibliográfica \\
            03/04/21                & Pesquisa Bibliográfica \\
            04/04/21                & Escrever Resultados no Relatório \\
            05/04/21                & Limpeza, Organização e Separação dos dados \\
            06/04/21                & Limpeza, Organização e Separação dos dados \\
            07/04/21                & Análise Exploratória dos dados \\
            08/04/21                & Análise Exploratória dos dados\\
            09/04/21                & Análise Exploratória dos dados\\
            10/04/21                & Escrever Resultados no Relatório\\
            11/04/21                & Implementar o Classificador Ingênuo de Bayes\\
            12/04/21                & Implementar o Classificador Ingênuo de Bayes\\
            13/04/21                & Implementar o Classificador Ingênuo de Bayes\\
            14/04/21                & Treinar o Classificador Ingênuo de Bayes\\
            15/04/21                & Consertar Bugs\\
            16/04/21                & Validar o Classificador Ingênuo de Bayes\\
            17/04/21                & Validar o Classificador Ingênuo de Bayes\\
            18/04/21                & Escrever Resultados Obtidos No Relatório\\
            19/04/21                & Discussão e Análise dos Resultados Obtidos Durante o Projeto\\
            20/04/21                & Produção dos Slides e Gravação do Vídeo\\
            21/04/21                & Produção dos Slides e Gravação do Vídeo\\
            22/04/21                & Entrega do Projeto\\
            
            \hline
        \end{tabular}
    \end{small}
\end{table}





\bibliography{mybib}
\nocite{*}
\bibliographystyle{IEEEtran}
\end{document}
