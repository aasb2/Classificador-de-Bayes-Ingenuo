\documentclass[conference]{IEEEtran}
\IEEEoverridecommandlockouts
% The preceding line is only needed to identify funding in the first footnote. If that is unneeded, please comment it out.
\usepackage{cite}
\usepackage{amsmath,amssymb,amsfonts}
\usepackage{algorithmic}
\usepackage{graphicx}
\usepackage{textcomp}
\usepackage{xcolor}
\usepackage{placeins}
\usepackage[hidelinks]{hyperref}
\usepackage{float}
\def\BibTeX{{\rm B\kern-.05em{\sc i\kern-.025em b}\kern-.08em
    T\kern-.1667em\lower.7ex\hbox{E}\kern-.125emX}}
\begin{document}

\title{Classificador Ingênuo de Bayes}

\author{\IEEEauthorblockN{Arthur Abrahão Santos Barbosa}
\IEEEauthorblockA{\textit{Universidade Federal de Pernambuco} \\
\textit{Centro de Informática}\\
Pernambuco, Brasil \\
aasb2@cin.ufpe.br}
\and
\IEEEauthorblockN{Filipe Samuel da Silva}
\IEEEauthorblockA{\textit{Universidade Federal de Pernambuco} \\
\textit{Centro de Informática}\\
Pernambuco, Brasil \\
fss8@cin.ufpe.br}
\and
\IEEEauthorblockN{Nigel Mendes de Lima}
\IEEEauthorblockA{\textit{Universidade Federal de Pernambuco} \\
\textit{Centro de Informática}\\
Pernambuco, Brasil \\
nml@cin.ufpe.br}

}

\maketitle





\section{Objetivos}
\subsection{Objetivo Geral}
Através da análise de algumas informações referentes a um indivíduo, usando um classificador ingênuo de Bayes, prever se o mesmo irá se inscrever em um depósito a prazo. 
\subsection{Objetivos Específicos}
\begin{itemize}
\item Compreender a implementação do classificador ingênuo de Bayes
\item Demonstrar a Importância do Aprendizado de máquina e suas aplicações
\item Investigar o uso do Aprendizado de máquina no marketing bancário
\end{itemize}
\section{Justificativa}
Este projeto foi escolhido com base na maneira organizada e completa que o conjunto de dados  foi disponibilizado e por sua afinidade em aplicar-se os conceitos existentes, o banco de dados pode ser encontrando do site Machine Learning Repository, com o nome de "Bank Marketing Data Set"\cite{b1}.


Sua função é prover dados sobre a possibilidade de um cliente aderir ou não o serviço prestado pela agência com base em testes com múltiplas entradas de dados e com duas saídas possíveis, sim ou não. Seu público alvo são principalmente bancos, qualquer área de estudo sobre comportamento social e estudos sobre aprendizagem de máquina.

\section{Base de Dados}

\section{Análise Exploratória dos Dados}

\section{Classificador Ingênuo de Bayes}

\section{Experimentos}

\section{Análise dos Resultados}

\section{Conclusões e Discussões}




\bibliography{mybib}
\nocite{*}
\bibliographystyle{IEEEtran}
\end{document}
